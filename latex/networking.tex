\documentclass{article}

\usepackage{titling}
\usepackage{geometry}
\usepackage{fontspec}
\usepackage{color}
\usepackage{xcolor}
\usepackage{tcolorbox}
\usepackage{fancyhdr}
\usepackage{hyperref}
\usepackage[skip=11pt]{parskip}
\usepackage[outputdir=/home/neil/.local/share/latex/output]{minted}

\definecolor{bg}{RGB}{22,43,58}

\tcbuselibrary{listings, minted, skins}
\tcbset{listing engine=minted}

\hypersetup{%
   colorlinks=true,
   linktoc=all,
   linkcolor=gray,
}

\setmainfont{LiberationSans}
\geometry{%
  margin=1in
}

\fancyhf{}
\renewcommand{\headrulewidth}{0pt}
\fancyfoot[R]{\thepage}
\pagestyle{fancy}
\fancypagestyle{plain}{\pagestyle{fancy}}

\renewcommand\maketitlehooka{\null\mbox{}\vfill}
\renewcommand\maketitlehookd{\vfill\null}

\title{Networking}
\author{Neil Kingdom}

\begin{document}

\begin{titlingpage}

\maketitle

\end{titlingpage}

\newpage

\tableofcontents

\newpage

\section{Introduction}

Whether you aspire to become a server administrator, IT technician, programmer, or even things which aren’t
tech related, networking is a massive part of our lives now, and it’s here to stay. The internet has a very
long history, which I will not be covering here, but that is okay because you do not need to understand how
the internet was conceived to understand how it works. If you do happen to be someone going into a technical
field, comprehending networking will likely be a requirement going forward in your field, since you will
innevitably have to deal with something network related eventually.

\section{Terminology}

Before we really begin, I’d like to start off right away by defining certain terms that will arise often as we
speak about networks, the internet, etc. Most of the definitions should be quite straight forward.

\begin{itemize}

\item{%
    \textbf{Network:} A network is a system of connections which are linked by some form of media. In computer
    science, networking refers to a system of connections between devices (often computers or servers) via
    wires, light, or wave transmissions.
}

\item{%
    \textbf{Internet:} The internet is a world-wide web (where the term www comes from in web addresses) of
    connections between routers. The internet is how you are able to connect to anyone or anything outside of
    your local household. Your Internet Service Provider (ISP) is in charge of linking your household to the
    internet.
}

\item{%
    \textbf{Intranet:} An intranet is similar to the internet, but on a much smaller scale. This is often
    found within large companies where employees might need access to resources about the company, but may
    still have restricted access to the internet. For example, when I worked at CRA, we had an intranet where
    we could search for information about particular branches or sects.
}

\item{%
    \textbf{Extranet:} A less common form of network are extranets. An extranet is simply an intranet which
    has been extended to public use. For instance, a company may want to keep certain things within their
    intranet private to employees, but other data may be acceptable for the public to view. The data that is
    acceptable for the public to view would belong to the companies extranet.
}

\end{itemize}

Often times, we talk about networks in terms of their size. This helps us get an idea about the scale of the
network. The following terms apply to network areas:

\begin{itemize}

\item{%
    \textbf{Peer to Peer (P2P):} Peer to peer is the smallest type of network possible – a connection between
    two devices. P2P is a specification of the broader Personal Area Network (PAN). For example, a phone
    connected to your car radio via bluetooth would be considered a PAN.
}

\item{%
    \textbf{LAN:} A very common term in networking, LANs are Local Area Networks. Your household is considered
    a LAN.
}

\item{%
    \textbf{MAN:} Metropolitan Area Networks often refer to a neighbourhood or large building like a hotel or
    retirement home.
}

\item{%
    \textbf{WAN:} Wide Area Networks are very large networks. Typically, WAN refers to the internet.
}

\item{%
    \textbf{SAN:} Storage Area Networks are often used to refer to server centers. A SAN is a connection of
    storage devices.
}

\item{%
    \textbf{CAN:} Campus Area Networks refer to a campus’ network like the one you may or may not be attending!
}

\end{itemize}

Note: Prepending a 'W' to any of the abbreviations mentioned above means "wireless". For example, a WLAN is
simply a wireless local area network.

\begin{itemize}

\item{%
    \textbf{Client:} The client always refers to the reciever of media. The client can request data, and often
    times send data, however, they are typically not the primary provider of the data.
}

\item{%
    \textbf{Host:} The formal definition of a host is any computer or device connected to a network. I do not
    really like this definition though. In computer science, the host often refers to the primary provider of
    data. Servers are almost always hosts because they provide data to the clients that connect to them.
}

\item{%
    \textbf{Server:} A physical device (often located in a remote location such as a special building) which
    stores data. Servers are most often responsible for storing information about web pages or customer data.
}

\end{itemize}

\section{Network Topology}

A topology diagram is used to help network engineers understand the details of how a particular network is
configured. Typically, a topology diagram has both a physical and logical counterpart. The physical diagram
shows the network engineer where devices reside physically within a building. This might include room numbers,
number of devices, etc. The logical diagram is usually the one we care about most. It shows an abstract
overview of the connections between each device. This helps us see how the network is split, which devices are
connected to which, and how the network connects to the internet if applicable.

Since we wont really be concerning ourselves with physical topology diagrams in this paper, I will show you the
various symbols that are used to represent various devices and media.

\end{document}
