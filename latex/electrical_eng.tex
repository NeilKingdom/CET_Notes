\documentclass{article}

\usepackage{titling}
\usepackage{geometry}
\usepackage{fontspec}
\usepackage{color}
\usepackage{xcolor}
\usepackage{tcolorbox}
\usepackage{fancyhdr}
\usepackage{hyperref}
\usepackage{tabularx}
\usepackage[american voltages]{circuitikz}
\usepackage[skip=11pt]{parskip}

\hypersetup{%
   colorlinks=true,
   linktoc=all,
   linkcolor=gray,
}

\setmainfont{LiberationSans}
\geometry{%
  margin=1in
}

\fancyhf{}
\renewcommand{\headrulewidth}{0pt}
\fancyfoot[R]{\thepage}
\pagestyle{fancy}
\fancypagestyle{plain}{\pagestyle{fancy}}

\renewcommand\maketitlehooka{\null\mbox{}\vfill}
\renewcommand\maketitlehookd{\vfill\null}

\title{Electrical Engineering}
\author{Neil Kingdom}

\begin{document}

\begin{titlingpage}

\maketitle

\end{titlingpage}

\newpage

\tableofcontents

\newpage

\section{Abstract}

\section{Introduction}

\section{Ohm's Law}

\section{Circuits}

\subsection{Parallel vs Series}

\section{Kirchhoff's Law}

\section{Alternative vs Direct Current}

\section{Passive components}

We'll now take a detour from some of the more theoretical and physical laws to get a grasp of practical hands
on concepts. This section will be focused on passive components, and the next will be focused on active
components. Electronic components exist within every eletronics device. Not all components can be classified as
a load because not all components necessarily consume power, however, they do affect the behaviour of current,
voltage, and resistance. Passive components are components which cannot control electric current utilizing a
secondary electrical signal. Active components, on the other hand (which we'll look at in a bit) can control
the current of the circuit. For example, a type of active component known as a transistor can amplify current
using a secondary signal. We'll begin by looking at arguably the most basic type of passive component: the
resistor.

\subsection{Resistors}

A resistor is perhaps one of the most common and simple components that makeup a circuit. Resistors add
resistance to the circuit, thus inversly reducing current. Resistors are typically placed in series next to a
load to reduce current draw. Resistors come in various shapes and sizes, as well as ratings. Since resistors
deal with resistance, they are measured in Ohms. The schematic symbol for a resistor can resemble either of
the symbols below.

\begin{circuitikz}
\draw (0,0) to[ R=$R_1$ ] (2,0);
\draw (3,0) to[ R=$R_2$, style={european resistors} ] (5,0);
\end{circuitikz}

Resistors are not polarized, meaning that they don't have a particular polarity as to how you orient them (no
positive/anode or negative/cathode leads). The style depicted on the left is more common in North America,
whereas you may see the style on the right used more in Europe.

Most resistors are fixed, meaning that they have a resistance rating in Ohms which is the resistance that
they'll stay at. Variable resistors, however, allow for tuning of the resistance between some range. A
potentiometer is a device with three leads, typically used for knobs and dials, which tunes the resistance
value and is usually intended to be operated by human intervention. A trim pot, on the other hand, is a
similar device, but one which is typically meant to be tuned once in the factory and then left untouched. The
schematic symbol for a variable resistor can appear as either the left or middle symbols and the right-hand
symbol represents the symbol for a trim pot.

\begin{circuitikz}
\draw (0,0) to[ pR  ] (2,0);
\draw (3,0) to[ vR ] (5,0);
\draw (6,0) to[ vR, tunable end arrow={Bar} ] (8,0);
\end{circuitikz}

The left-hand symbol with the arrow pointing down is sometimes called a tap, which represents the position of
the physical slide that affects its resistance value. Note that we could have used the European symbol for the
resistor rather than the North American symbol for either of these, as well as the ones I'll show in the
future, but I will be sticking with the North American version for future reference.

Some more types of variable resistor which are a bit more niche, but still used in certain applications
include the thermistor, which is a resistor that changes resistance with temperature; photoresitor, which is
photosensitive and fluxuates with light; as well as the varistor, which changes resistance according to
voltage. To give some examples of where these are used: thermistors are used in 3D printers to sense the
temperature of the hotend and varistors are used for surge protection, typically in conjunction with a fuse to
mitigate the damage done to a circuit when high levels of voltage are applied. From left to right, I've laid
out the thermistor, photoresistor, and varistor schematic symbols for you.

\begin{circuitikz}
\draw (0,0) to[ thR ] (2,0);
\draw (3,0) to[ ldR ] (5,0);
\draw (6,0) to[ mov ] (8,0);
\end{circuitikz}

To provide some more nuance, the schematic symbol I've actually used for the varistor is the symbol for a
metal-oxide varistor A.K.A. a MOV, which is likely the most common type you'll see. Another thing to note is
that thermistors come in two forms: Positive Termperature Coefficient (PTC) and Negative Temperature
Coefficient (NTC). PTC means that resistance increases as temperature increases, and vice versa for NTC. Also
note the two diagonal arrows pointing towards the photoresistor. These represent photons or light, and can be
used for other components to signify that the component is photosensitive. Arrows pointing the opposite
direction indicate that the component is light-emitting, which we'll come to see with LEDs.

\subsection{Capacitors}

Similar to resistors, capacitors are another simple, yet crucial and commonly used component within electronic
circuits. A capacitor, as the name suggests, has some capacitance for current, which builds incrementally
before eventually discharging. Capacitors are typically used to dampen or smooth electrical impedance or
sudden spikes in current due to various factors. They are also commonly used in timer circuits or placed in
series with switches to reduce the risk of what's known as bouncing, whereby a switch or button registers
multiple inputs even though the operator only intended to input one press.

To get a bit more technical, capacitors are made up of two metal plates which sandwich the dielectric. The
dielectric is a material or fluid which stores the energy between the voltage differential. The dielectric is
typically either a fluid containing electrolyte, which is where we get the term elecrolytic capacitors, or it
will be ceramic, hence the term ceramic capacitors. These two have pros and cons which I won't get into such
as size, heat tolerance, etc. The primary difference, however, is that elecrolytic capacitors have a polarity,
whereas ceramic capacitors do not. Sometimes we refer to non-polarized capacitors as "fixed" capacitors.
Similar to resistors, we can also have variable capacitors which can alter their capacitance. Below you'll see
the symbols for fixed (left) and polarized capacitors (middle and right).

\begin{circuitikz}
\draw (0,0) to[ C=$C_1$ ] (2,0);
\draw (3,0) to[ eC=$C_2$ ] (5,0);
\draw (6,0) to[ cC=$C_3$ ] (8,0);
\end{circuitikz}

There exist other variations for electrolytic capacitors as well, though they are all fairly similar to the
ones depicted above. A variable resistor, as well as any component with a variable quantity will use the same
arrow that we used for variable resistors, and likewise for the trimmer capacitor, which are both depicted
below.

\begin{circuitikz}
\draw (0,0) to[ vC ] (2,0);
\draw (3,0) to[ vC, tunable end arrow={Bar} ] (5,0);
\end{circuitikz}

Note that for several variable components, we sometimes see them being "ganged" A.K.A. "linked", meaning that
the changes applied to one component will cause the same effect to occur in the other. We denote that two
variable components are ganged together using a dashed line to connect the bottom of the arrow as illustrated
below for the two ganged variable capacitors.

\begin{circuitikz}
\draw (0,0) to[ vC=$VC1A$, n=A ] (1.5,0);
\draw (2,0) to[ vC=$VC1B$, n=B ] (3.5,0);
\draw[dashed] (A.wiper) -- (B.wiper);
\end{circuitikz}

\subsection{Inductors}

\subsection{Transformers}

\section{Active Components}

\subsection{MOSFETs}

\subsection{Transistors}

\section{Rectifiers}

\section{Electromagnatism}

\subsection{Eddy Currents}

\section{Operational Amplifiers}

\section{Through-Hole vs Surface Mount}

\section{Semiconductors}

\section{Integrated Circuits}

\subsection{IC Package Types}

\subsection{System on a Chip}

\section{Pulse-Width Modulation}

\section{Buck Converters}

\section{Motors}

\subsection{DC Brush Motor}

\subsection{Brushless DC Motor}

\end{document}
