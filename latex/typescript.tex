\documentclass{article}

\usepackage{titling}
\usepackage{geometry}
\usepackage{fontspec}
\usepackage{color}
\usepackage{xcolor}
\usepackage{tcolorbox}
\usepackage{fancyhdr}
\usepackage{hyperref}
\usepackage[skip=11pt]{parskip}
\usepackage[outputdir=/home/neil/.local/share/latex/output]{minted}

\definecolor{bg}{RGB}{22,43,58}

\tcbuselibrary{listings, minted, skins}
\tcbset{listing engine=minted}

% TS codeblocks
\newtcblisting{tslst}{%
   listing only,
   minted language=typescript,
   minted style=monokai,
   colback=bg,
   enhanced,
   frame hidden,
   minted options={%
      tabsize=4,
      breaklines,
      autogobble
   }
}

\hypersetup{%
   colorlinks=true,
   linktoc=all,
   linkcolor=gray,
}

\setmainfont{LiberationSans}
\geometry{%
  margin=1in
}

\fancyhf{}
\renewcommand{\headrulewidth}{0pt}
\fancyfoot[R]{\thepage}
\pagestyle{fancy}
\fancypagestyle{plain}{\pagestyle{fancy}}

\renewcommand\maketitlehooka{\null\mbox{}\vfill}
\renewcommand\maketitlehookd{\vfill\null}

\title{TypeScript}
\author{Neil Kingdom}

\begin{document}

\begin{titlingpage}

\maketitle

\end{titlingpage}

\newpage

\tableofcontents

\newpage

\section{Abstract}

I've titled this document 'TypeScript' because that is what it primarily concerns. Unlike most of my other
documents which specifically target the features of a particular language, this document will be a bit more
broad in scope, opting to stray away from mere TypeScript to take a look at the history of web libraries and
frameworks. This document is meant to be read as a companion to my other document 'Web Applications'. I felt
that there was an exceedingly large amount of content, so some of it has been moved here. I hope that by taking
a look into some of the history of writing web applications, you're knowledge of TypeScript will be bolstered
all the more.

\section{Introduction}

TypeScript is a bit of a rare specimen, because it isn't \emph{really} its own language. What I mean by that is
that it is really just syntactic sugar on top of JavaScript. Because of this, there don't currently exist any
interpreters to my knowledge which interpret TypeScript by itself. Instead, it is common to compile TypeScript
into JavaScript first, and then use an existing interpreter to run that. This makes sense because most things
are already compatible with JavaScript, and trying to make everyone adopt a secondary language would likely not
go over well. As mentioned in the abstract, in this document, we'll be covering TypeScript, but also some other
libraries and frameworks to give a better idea of TypeScript's role within the industry.

\section{JavaScript}

Alongside HTML and CSS, JavaScript (a.k.a. JS) makes up the third person in the web trinity. JavaScript is a
dynamically typed language, meaning that it does not strongly enforce types. It is also an interpreted language
which depends on a Just-in-Time (JIT) compiler to intepret it at runtime, similar to Python. The language is
inherently single-threaded, which is why it relies heavily on promises and futures and asynchronous methods.
JavaScript was created in just 10 days by Brendan Eich who was working for Netscape at the time, but is now one
of the most utilized languages on the web. JavaScript conforms to something known as the ECMAScript (a.k.a. ES)
standard, which is also the standard used for JScript and ActionScript (languages similar to JavaScript, but
not identical).

JavaScript can be considered a scripting language. It's original intent was to manipulate elements of the
Document Object Model (DOM). As web 2.0 roled around, it became apparent that we needed a method of handling
more complex logic within our web apps. Considering JavaScript was already supported in most browsers, and that
it was a perfectly capable language, it became the standard for interfacing with the backend and handling most
of the application's logic. Considering TypeScript is essentially just JavaScript with types, I won't be
covering JavaScript on its own, but learning TypeScript should equip you with everything you need to understand
JavaScript.

\section{JQuery}

JQuery used to be a very popular library for JavaScript. JQuery is still used on many websites, and still has
certain niche applications. Back in the day, we used to use JavaScript to operate on the DOM directly. We would
select an element using something like getElementById() and then operate on said element. JQuery provided a
wrapper which made selecting elements even easier, and also provided ways to easily manipulate the CSS classes
associated with elements. Nowadays, with page rehydration, we avoid operating on the DOM directly, since a) it's
very error-prone, and b) it requires reloading the entire DOM any time a change needs to be displayed, rather
than simply refreshing the element which was updated.

You can tell you're looking at JQuery code when you see a bunch of \$ signs being used everywhere. The \$ sign
expands to the jQuery() function. The jQuery() function (or just \$()) has multiple overloads. It returns a
jQuery object, which differs from a standard JavaScript object. We can grab elements by selector or element.
For example:

\begin{tslst}

$( "div.foo" ).on( "click", function() {
    $( "span", this ).addClass( "bar" );
});

\end{tslst}

The example above grabs any div element with class "foo", then sets its onClick handler to an anonymous
function which adds the class "bar" to any span elements within the outer div element. jQuery is still a good
library if you really do require something that can manipulate the DOM directy. For example, it might be a
useful tool when writing web scrapers that need to extract specific data. It is not advisable that it be used
when creating web applications, however.

\section{The V8 Engine}

As mentioned earlier, JavaScript runs in the browser. In Chrome, this is accomplished by using the V8 engine.
The V8 engine is Google's JavaScript engine, which contains its own JIT compiler for JS. The V8 engine is
written in C++. It implements the ECMAScript standard, as well as Web Assembly (WASM).

\section{Node.js}

In the previous section, we seen how the V8 engine could interpret JavaScript from within the browser. Only
being able to run JavaScript within the browser was a bit of a limitation of the language, however. Enter
node.js, a runtime environment which utilizes the V8 engine to allow JavaScript to be able to be ran from
anywhere. Node.js is sort of to JS what the JVM is to Java. So long as Node.js can be installed on the user's
native platform, then it can interpret JS code. On a bit of a different note, the Electron framework utilizes
Node.js to provide users with tools for building traditional GUI desktop applications. Electron is often
heavily criticized for utilizing so much memory since it's not only built on top of multiple layers of
abstraction, but also requires JIT compilation. Still, many apps such as the Atom and VSCode IDEs.

node.js works similarly to Python, in that you can run the command on its own and you'll enter a sandboxed
environment where you can execute JS statements in real time. This is pretty nice when you just want to see
what something evaluates to quickly without setting up a whole project.

\section{package.json}

node.js introduced the package.json file for managing node projects. node.js will search for this file in order
to find metadata about the project's structure. This is essentially equivallent to Java's MANIFEST.md file or
Rust's cargo.toml file. package.json has a couple properties that can and should be overridden when creating a
project in JS or TS. These include the following:

\begin{itemize}

\item \textbf{name:} Sets the name of the project.

\item \textbf{version:} Sets the version number for the project.

\item \textbf{license:} Sets the type of license used for the project.

\item \textbf{description:} Sets a brief description about the project.

\item \textbf{keywords:} Sets a brief description about the project.

\item \textbf{main:} The file containing the entry point of the program.

\item \textbf{repository:} Used for specifying metadata about an upstream repository e.g. GitHub.

\item \textbf{scripts:} Creates custom build actions similar to rules in a Makefile.

\item \textbf{dependencies:} A list of dependencies and their expected versions. Usually managed by npm.

\item{%
    \textbf{devdependencies:} A list of developer dependencies and their expected versions. Usually managed by
    npm.
}

\end{itemize}

\section{Node Package Manager}

Node Package Manager (NPM) is the official package manager for node.js. npm is a pretty straightforward package
manager, though there are a couple of things to note before you go out and start installing a bunch of packages.
First of all, npm is notorious for having vulnerabilities in its packages since anyone can easily upload their
libraries to the upstream repo, and because the web is such a high priority target for hackers. npm will mark
vulnerable packages as such, but this will only occur after the vulnerabilities have been reported and
confirmed.

To find a package by keyword, use npm search [search terms]. If a term begins with a forward slash, it shall be
interpreted as a regex pattern. When it comes to installing packages, you have a few options that you might
want to consider. Before I get carried away, note that you can run npm help <subcommand> to bring up a man
page containing additional information about that subcommand. For npm install, the most important option is
the -g or --global flag, which will install the package system-wide. The default is to install the package in
the user's current directory in a folder called node\_modules. As mentioned in the previous section, npm will
utilize package.json, alongside node.js. npm primarily utilizes this file to track dependencies and dependency
versions. package.json may contain three different subsections for packages. These are: production,
developement, and optional dependencies. These subsections correspond to the -P, -D, and -O flags,
respectively, when running the npm install subcommand. -P is the default option. These help distinguish which
packages are actually critical for the application, and which ones are used for developer tools or environment
setup. Running npm install by itself will synchronize any packages located in package.json, meaning that any
dependencies which are out of date or which are not installed will either be upgraded or installed,
respectively. If your packages become corrupt, it's easiest to delete the node\_modules directory and simply run
npm install. It should also be noted that most npm subcommands have aliases, so npm i and npm install are
equivallent.

You can run npm init to intialize the current directory. Unless the -y option is used, this will prompt you to
fill out several of the typical properties in the package.json file. npm init essentially is just a fancy
script for generating the package.json file, and that's about it.

We also mentioned in the previous section that we could create custom build actions in package.json within the
scripts property. The scripts property can be set to an object containing one or more rules. These rules can
be executed using the npm run subcommand. npm defines a couple default rules which can be invoked directly
instead of having to use the run subcommand. These include: test, start, restart, and stop. So in other words,
rather than invoking npm run start, we could just say npm start instead; but only for the aforementioned
aliases, otherwise you must use npm run <command>. Here's an example:

\begin{tslst}

    "scripts": {
        "start": "node index.js",
        "test": "jest",
        "custom": "./run_script.sh"
    }

\end{tslst}

\section{Node Package Executor}

The Node Package Executor (NPX) is bundled with newer releases of npm. If you do not have it installed, you can
install it globally via npm e.g. npm i -g npx. The npx command is capable of executing packages installed with
npm from the command line. This allows devs to write CLI tools in JS, upload them to the npm repository and
then anyone using that CLI tool can execute it using npx.

\section{Managing the Node.js Release Version}

We've seen how to manage packages for a project using npm, but what about managing the release of node.js
itself? For this, we can use another CLI tool called the Node Version Manger (NVM). nvm is the simplest
way of switching the version of node.js that your system is using. This is very helpful when maintaining
projects that use different node.js versions. You can list out the node.js versions available to you by
running nvm ls-remote. nvm install --lts will install the latest Long Term Release version, nvm install node
to install the latest version available, or nvm install <version> with the specific version you want. Likewise
running nvm use with any of the same options for install will activate the appropriate node js version on the
system.

\section{Alternative Package Managers}

\subsection{pnpm}

\subsection{Yarn}

\section{TypeScript as a Dependency}

All this talking and yet we are just now getting to the bread and butter of using TypeScript. As mentioned, TS
isn't really its own language per se. For this reason, we can actually just install it as a developer
dependency for our project e.g. npm i -D typescript. TS should now appear as a dependency in the node\_modules
directory. If we do an ls on node\_modules/typescript/bin, we can find two useful tools for working with TS:
tsc and tsserver. Let's look at each of these individually.

\subsection{The TypeScript Compiler}

The TypeScript Compiler (TSC) is the official compiler for converting from TS to JS.

\subsection{TS Server}

\section{React}

NOTE: Update (react is just a library, not a framework)
React is arguably the most popular framework on the market in current year. This is mostly in part thanks to
its backing by Facebook, and its more mature lifetime (frameworks such as Angular, Vue JS, and Svelte are all
significantly more recent). One thing that I think confuses new web developers learning React is its use of
JSX. JSX is truly a Frankenstein mishmash of JavaScript and HTML, which makes it sometimes difficult to
distinguish. For instance, it is possible using JSX to assign an HTML element to a JavaScript variable.

\section{Babel.js}

Babel.js is a JavaScript compiler. It's actually more like a transpiler, meaning that it converts high level
code into code of essentially the same complexity. Babel.js is primarily used for compiling current versions
of JS into older versions for browsers that don't support later versions. Babel.js is also used for compiling
JSX and React into JS, which is likely what it's most often used for.

\section{JSX}

\begin{tslst}

const element = <h1>This line of code uses JSX</h1>

\end{tslst}

What React fundamentally allows us to do is create custom JSX elements/components which look like HTML tags
and accept attributes in the same manner. This powerful heirarchical pattern allows us to build components of
off other pre-existing components, which makes development quicker and less repetitive. Practically speaking,
components are created by returning a block of JSX code from a function. A new tag is then generated based off
the name of the function. React components must use Pascal casing, which distinguishes them from regular
functions. For example, we can create a custom button component with the following code:

\begin{tslst}

const CustomButton = (): JSX.Element => {
    const sayHello = () => {
        alert('Hello, World');
    }

    return (
        <div>
            <button
                style={ color: 'red' }
                onClick={ sayHello }>
                My Custom Button
            </button>
        </div>
    );
};

export default CustomButton;

\end{tslst}

Then elsewhere, we can import this component and use it like so (note that App() is the entry point for
React):

\begin{tslst}

import CustomButton from 'path/to/file/CustomButton';

function App(): JSX.Element => {
    ...
    <CustomButton />
    ...
};

export default App;

\end{tslst}

We can escape JSX and insert JavaScript code using curly braces. This is what we did for the style and
onClick attributes in the former example. Likewise, any comments must first be escaped like so:

\begin{tslst}

{ /* A comment in JSX */ }

\end{tslst}

React uses hydration to update individual components, rather than reloading the entire DOM each time the state
of a component changes. This is, naturally, much faster than attempting to manipulate individual elements in
the DOM, which is what something like jQuery does. Once again, looking at the practical application in terms of
how this is achieved, React uses something called hooks (essentially a glorified name for callback functions)
to update Components when the state is changed. There are many kinds of hooks, which we'll look at shortly, but
the relevant kind that we're interested in for updating components on state change are state hooks. You can
tell that a function is a hook in React if it begins with the word 'use'. State hooks are created using the
useState() function. What this does is return a list with exactly two items. The first item it returns will be
the variable that will maintain state, and the second item will be a callback function that can be used to
update the state of this variable. The idea is to keep the state variable as immutable i.e. we should not
update the state variable directly, but rather indirectly, via the callback. Typically, we name the state
variable some relevant name, and we name the callback the same thing, but prepended with 'set', akin to a
setter in OOP languages. One final note before I completely lose you: the useState() hook can take a parameter
(in TS this would be of type any) and assign that value as the initial state for the state variable. Let's
look at this in code:

\begin{tslst}

function App(): JSX.Element {
    const [counter, setCounter] = useState(3);
    return (
        <div>
            <h1>{ counter }</h1>
            <button onClick={ () => { setCounter(counter + 1); } }>
                Increment counter
            </button>
        </div>
    );
}

export default App;

\end{tslst}

\section{Hooks}

React uses something called hooks, which are just a fancy word for callbacks/function pointers that do a thing.
In React, all hooks begin with the word use, followed by the actual hook name. We'll look at some commonly
used hooks and why you'd want to implement them in your code. Hooks are idiomatic in React because they
typically take advantage of React's page hydration, where lone elements can be updated apart from the rest of
the DOM.

\subsection{useState}

The useState hook is by far the most common. It is used for creating a state variable. We can give this state
variable an initial value, which is what the component will render on its initial load. Using the setter that
useState provides, we can also update the state variable, thus in turn updating any components which rely on
it. As with most hooks, useState returns an array containing both the state variable and setter for said
variable. We tend to use array deconstruction to perform multiple assignment as seen in the code snippet below:

\begin{tslst}

import React, { useState } from 'react';

const [state, setState] = useState<string>('Initial value');

\end{tslst}

It is common to see this type of code being used when working with React. The setter is almost always given the
same name as the state variable but is prepended with 'set', as is the normal convension for getters/setters.

\subsection{useEffect}

\subsection{useContext}

\subsection{useDeferredValue}

\subsection{useId}

\section{ESLint}

The ECMA Script linter (eslint) is a static code analysis tool which enforces pre-defined code styling and
standards. ESLint is not really intended to catch semantic errors so much as it is meant to catch poor code
habits. It can enforce things like spacing, indentation, etc.

\section{Async Await}

JavaScript introduced the concept of promises and futures. I've spoken about promises and futures in many of my
other programming language documents, but to summarize, promises and futures are an asynchronous idiom. Async
code does not run parallel like with multithreading, but it does run concurrently. Async portions of code are
categorized into coroutines. The cpu is able to perform context switching to jump between coroutines and
process code without blocking for a specific line to finish execution. A function marked with the async
keyword must return a Promise<T>. Let's first look at how we used to use promises and futures before moving on
to the async await syntax so that we can get a better idea of what's happening behind the scenes.

Assume you have some function that pulls data from a database and might take an extended period of time to run.
We can simulate this wait time using setTimeout(). Let's take a look at the following code snippet, which will
continously run until either the timeout expires, or Math.random() chooses a number greater than 0.5.

\begin{tslst}

function fetchData(url: string): Promise<string> {
    return new Promise((resolve, reject) => {
        setTimeout(() => {
            if (Math.random() > 0.5) {
                resolve("Data fetched successfully");
            } else {
                reject(new Error("Network error"));
            }
        }, 100);
    });
}

\end{tslst}

In the code above, we return a string wrapped in a Promise object. When using the new keyword when constructing
a Promise object, it will expect a function with the resolve and reject parameters. These are functions which
are used to specify whether a task completed successfully (resolve) or failed (reject). When invoking an async
function, it is common to chain the then() and catch() methods rather than using try-catch blocks.

\begin{tslst}

fetchData("My SQL query")
    .then((response) => {
        console.log(response);
    })
    .catch((error) => {
        console.error(error);
    });

\end{tslst}

Unfortunately promises/futures end up becoming quite messy and cumbersome to read. The async/await keywords
provide syntactic sugar over top of promies/futures to make code read as though it it's procedural, even if
it's not.

\end{document}
