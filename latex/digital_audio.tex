\documentclass{article}

\usepackage{titling}
\usepackage{geometry}
\usepackage{fontspec}
\usepackage{color}
\usepackage{xcolor}
\usepackage{tcolorbox}
\usepackage{fancyhdr}
\usepackage{hyperref}
\usepackage[skip=11pt]{parskip}
\usepackage[outputdir=/home/neil/.local/share/latex/output]{minted}

\definecolor{bg}{RGB}{22,43,58}

\tcbuselibrary{listings, minted, skins}
\tcbset{listing engine=minted}

% C codeblocks
\newtcblisting{clst}{%
   listing only,
   minted language=c,
   minted style=monokai,
   colback=bg,
   enhanced,
   frame hidden,
   minted options={%
      tabsize=4,
      breaklines,
      autogobble
   }
}

\hypersetup{%
   colorlinks=true,
   linktoc=all,
   linkcolor=gray,
}

\setmainfont{LiberationSans}
\geometry{%
  margin=1in
}

\fancyhf{}
\renewcommand{\headrulewidth}{0pt}
\fancyfoot[R]{\thepage}
\pagestyle{fancy}
\fancypagestyle{plain}{\pagestyle{fancy}}

\renewcommand\maketitlehooka{\null\mbox{}\vfill}
\renewcommand\maketitlehookd{\vfill\null}

\title{Digital Audio}
\author{Neil Kingdom}

\begin{document}

\begin{titlingpage}

\maketitle

\end{titlingpage}

\newpage

\tableofcontents

\newpage

\section{Abstract}

Similar to my computer graphics notes, I wrote this document partially out of my own self-interest, but also
to provide you with some foundational knowledge for the mechanics and maths required to compute sound. Audio
in and of itself is actually a huge domain, especially when it comes to computer science. I hope that, seeing
as this is not likely something you were taught in school or elsewhere, that this is an engaging topic.

\section{Introduction}

\end{document}
