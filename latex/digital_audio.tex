\documentclass{article}

\usepackage{titling}
\usepackage{geometry}
\usepackage{fontspec}
\usepackage{color}
\usepackage{xcolor}
\usepackage{tcolorbox}
\usepackage{fancyhdr}
\usepackage{hyperref}
\usepackage[skip=11pt]{parskip}
\usepackage[outputdir=/home/neil/.local/share/latex/output]{minted}

\definecolor{bg}{RGB}{22,43,58}

\tcbuselibrary{listings, minted, skins}
\tcbset{listing engine=minted}

% C codeblocks
\newtcblisting{clst}{%
   listing only,
   minted language=c,
   minted style=monokai,
   colback=bg,
   enhanced,
   frame hidden,
   minted options={%
      tabsize=4,
      breaklines,
      autogobble
   }
}

\hypersetup{%
   colorlinks=true,
   linktoc=all,
   linkcolor=gray,
}

\setmainfont{LiberationSans}
\geometry{%
  margin=1in
}

\fancyhf{}
\renewcommand{\headrulewidth}{0pt}
\fancyfoot[R]{\thepage}
\pagestyle{fancy}
\fancypagestyle{plain}{\pagestyle{fancy}}

\renewcommand\maketitlehooka{\null\mbox{}\vfill}
\renewcommand\maketitlehookd{\vfill\null}

\title{Digital Audio}
\author{Neil Kingdom}

\begin{document}

\begin{titlingpage}

\maketitle

\end{titlingpage}

\newpage

\tableofcontents

\newpage

\section{Abstract}

Similar to my computer graphics notes, I wrote this document partially out of my own self-interest, but also
to provide you with some foundational knowledge for the mechanics and maths required to compute sound. Audio
in and of itself is actually a huge domain, especially when it comes to computer science. I hope that, seeing
as this is not likely something you were taught in school or elsewhere, this will be an engaging topic.

\section{Introduction}

Music has been around for quite some time, and as it turns out, people seem to like it. What's interesting is
that we can very ellegantly describe the noises we perceive using physics. As you're most probably aware if
you've passed highschool, noise is caused by the vibrating of air molecules at a certain frequency. The
cohclea i.e., eardrum, vibrates in turn, which transmits a signal to the brain and so forth. Music happens to
be deeply rooted in math, whether it be musical scales, chords, time signatures, etc. We won't really be
getting into musical theory here, because I'm no musical expert, but I can hopefully help give you a different
perspective on music and how we learned to process it digitally.

\section{Amplitude and Pitch}

If you're not comfortable with maths involving sinusoidal waves, I suggest you learn that first and come back
to this later, because we will be dealing with sin waves a lot. I will try to explain these concepts at a
pretty basic level, but I will at least be assuming you're familiar with the unit circle, radians, pi, and these
sorts of things.

Assuming you are familiar with sin waves, let me help refresh your memory a bit. In audio processing, amplitude
refers to volume. The higher the amplitude, the louder the sound. When dealing with sinusoidal waves, increasing
the amplitude corresponds to increasing the distance between the peaks and valleys of the wave (to put it in
laymans terms). Frequency in audio refers to the pitch of the tone i.e. how high pitched or deep the tone
sounds. Notes were created on the basis of fractional ratios called the diatonic scale, which use intervals
of perfect fifths. This is how we get the seemingly odd arrangement of white and black keys on a piano. In
short though, notes are just specific frequencies which are each some equal ratio apart.

\section{Frequency}

As we discussed, frequency corresponds to pitch. Frequency in the mathematical sense is a measurement which
uses units called Hertz (Hz). It is the measure of cycles per second. Frequency is used to measure many things
apart from just pitch, as it can also be used to measure the oscillation of an electronic signal generator for
example.

Imagine that we have the function f(x) = sin(x). Imagine now that x is the number of seconds that have
transpired (maybe it would have been better to use t instead of x, but you get the point). This poses a problem,
because after a single cycle, x will be 2pi. Since we work with Hertz in audio processing, the first step is
to alter the frequency of our sin wave such that each cycle takes one second. We do this by multiplying x by
2pi, which normalizes the frequency e.g. f(x) = sin(x * 2pi).

\subsection{Sample Rate}

\section{Replicating Sound Electronically}

\subsection{Wave Carriers}

\subsection{Modulation}

\subsection{Converting Between Analog and Digital}

\section{Chip Tunes}

\subsection{Square Waves}

\subsection{Sawtooth Waves}

\subsection{Triangle Waves}

\subsection{Sawtooth Waves}

\section{Sinks/Sources}

\section{Channels}

\section{Mixing}

\section{Fourier Series}

\section{Fourier Transform}

\section{Fast Fourier Transform (FFT)}

\section{MIDI}

\section{Envelopes}

\end{document}
